
\documentclass[letterpaper,12pt]{article}
\usepackage{tabularx} % extra features for tabular environment
\usepackage{amsmath}  % improve math presentation
\usepackage{float}
\usepackage{pdfpages}

\usepackage{multicol}
\usepackage{graphicx} % takes care of graphic including machinery
\graphicspath{ {./figures/} }
%\usepackage[margin=1in,letterpaper]{geometry} % decreases margins
%\usepackage{cite} % takes care of citations
\usepackage[final]{hyperref} % adds hyper links inside the generated pdf file
\hypersetup{
	colorlinks=true,       % false: boxed links; true: colored links
	linkcolor=blue,        % color of internal links
	citecolor=blue,        % color of links to bibliography
	filecolor=magenta,     % color of file links
	urlcolor =blue         
}
\usepackage[margin = 1in,headsep=0.5cm,headheight=2cm,letterpaper]{geometry} 

\usepackage{fancyhdr}
\pagestyle{fancy}
%\cfoot{center of the footer!}
\renewcommand{\headrulewidth}{0.1pt}



\begin{document}
\thispagestyle{empty}

\title{Fall 2022 EE301  \protect\\ Homework 1}
\author{Ahmet Akman 2442366 \protect\\ Hasan Oğuzhan Gök 2443083 }
\date{\today}
\maketitle
\tableofcontents
%\begin{abstract}
%abstract
%\end{abstract}
\section{Question 1 Solution}
Input/output relation: \(y_1(t)= x(t)cos(2 \pi f_0 t) \)
\\ Properties:

\begin{itemize}
    \item \textbf{Memory:} The system is memoryless since the output is only depend on the current t value.
    \item \textbf{Linearity:} The system is linear it can be shown as follows: \begin{equation}\begin{split}
        \underline{input} &\Rightarrow \underline{output} \\
        x_1(t)  & \Rightarrow x_1(t) cos(2\pi f_0 t) \\
        a x_2(t) & \Rightarrow  a x_2(t) cos(2\pi f_0 t) \\
        x_1(t) +  a x_2(t)  &\Rightarrow x_1(t) cos(2\pi f_0 t) + a x_2(t) cos(2\pi f_0 t)      \end{split}
    \end{equation} 
    So the linearity holds.
    \item \textbf{Causality:} Since the system is memoryless it can be said that the system is causal.
    \item \textbf{Time-Invariance:} The system is time-varying since a time shift in the input does not result in same time shift in the output. \begin{equation}
        \begin{split}
            Let x_2(t) &= x_1(t-t_0) \\
            y_1(t) &= x_1(t) cos(2 \pi f_0 t) \\ 
            y_2(t) &= x_2(t) cos(2 \pi f_0 t)\\
            y_1(t-t_0) &=    x_1(t-t_0) cos(2 \pi f_0 (t-t_0))\\
            \text{However  } y_2(t) & \neq y_1(t-t_0)
        \end{split}
    \end{equation}
    \item \textbf{Stability:} Since \(-1 \leq cos(alpha) \leq 1\) if the x(t) is bounded the y(t) is bounded as well. So the system is stable.
\end{itemize}
Input/output relation: \(y_2(t)= c_1 x(t) + c_2 x^2(t) \)
\\ Properties:
\begin{itemize}
    \item \textbf{Memory:} The system is memoryless since the output is only depend on the current t value.
    \item \textbf{Linearity:}The system is non-linear and it can be shown as follows: \begin{equation}\begin{split}
        \underline{input} &\Rightarrow \underline{output} \\
        x_1(t)  & \Rightarrow c_1 x_1(t) + c_2 x_1^2(t) \\
        a x_2(t) & \Rightarrow  c_1 a x_2(t) + a^2 c_2 x_2^2(t) \\
        x_1(t) +  a x_2(t)  & \not\Rightarrow   c_1 (x_1(t) + a x_2(t)) + c_2 (x_1^2(t) + a^2 x_2^2(t) )
          \end{split}
    \end{equation} 
    \item \textbf{Causality:} Since the system is memoryless it can be said that the system is causal.
    \item \textbf{Time-Invariance:} The system is time invariant and it can be shown as follows;
    \item \begin{equation}
        \begin{split}
            \text{Let } x_2(t) &= x_1(t-t_0) \\
            y_1(t) &= c_1 x_1(t) + c_2 x_1^2(t) \\ 
            y_2(t) &= x_2(t) + c_2 x_2^2(t) \\
            y_1(t-t_0) &=    c_1 x_1(t-t_0) + c_2 x_1^2(t-t_0) \\
            \text{Then  } y_2(t) &= y_1(t-t_0)
        \end{split}
    \end{equation}
    \item \textbf{Stability:} System is stable, that is if the input of \(x(t)\) is bounded output is also bounded.
\end{itemize}


Input/output relation: \(y_3(t)= x(t)+4 \)
\\ Properties:
\begin{itemize}
    \item \textbf{Memory:} The system is memoryless since it just shifts the input signal.
    \item \textbf{Linearity:} Let  \(x(t)\) be equal to 0. The output is not equal to the 0. So the system is not linear.
    \item \textbf{Causality:} Since the system is memoryless it is causal.
    \item \textbf{Time-Invariance:} Any time shift on the input signal results in same amount of time shift on the output signal. So the system is time invariant.
    \item \textbf{Stability:} The system is stable, as the output is just the time shifted version of the input. 
\end{itemize}

Input/output relation: \(y_4(t)= x(t/3) \)
\\ Properties:
\begin{itemize}
    \item \textbf{Memory:}  The system is with memory since it squeezes the input signal.
    \item \textbf{Linearity:} The system is linear.
    \item \textbf{Causality:} Since the output may depend on different time instant other than the current t the system is not causal.
    \item \textbf{Time-Invariance:} As the input signal is squeezed the time shift in the input may not result in same amount of shift in the output.
    \item \textbf{Stability:} The system is stable as the magnitude stays for this system always.
\end{itemize}

Input/output relation: \(y_5(t)= t x(t+5) \)
\\ Properties:


\begin{itemize}
    \item \textbf{Memory:} The system is with memory because of the t+5 term.
    \item \textbf{Linearity:} The system is linear because the result stays linear if we multiply the input with a constant and sum it with another input. 
    \item \textbf{Causality:} The system is not causal because of the fact that the value of the input at t+5 time instant is used.
    \item \textbf{Time-Invariance:} The system is time-varying. The reason can be expressed as follows: \begin{equation}
        \begin{split}
             x_2(t+5) &= x_1(t+t_0+5) \\
             y_2(t) &\neq y_1(t) 
        \end{split}
    \end{equation}
    \item \textbf{Stability:} System is not stable consider the case where time is at infinity.
\end{itemize}

Input/output relation: \(y_6(t)= u(x(t)) \)
\\ Properties:
\begin{itemize}
    \item \textbf{Memory:} System is memoryless since the output depend only on the current input.
    \item \textbf{Linearity:} The system is not linear. Consider the case \(t = 0^- \and t = 0^+\)
    \item \textbf{Causality:} Memoylessness implies the system is causal.
    \item \textbf{Time-Invariance:} The system is time invariant. The shift in input leads same amount of shift in output.
    \item \textbf{Stability:} The system is stable. The outputs are bounded as long as inputs are bounded.
\end{itemize}



\section{Question 2 Solution}
\subsection{a)}
To be able to find the output signals the input signals are convolved with the inputs. The outputs are obtained with the computations given below;
\begin{equation}
    \begin{split}
        h(t) \ast 1 = \int_{-\infty}^{\infty} \alpha^{\tau-t} u(\tau-t)  \,d\tau  
    \end{split}
\end{equation}
\subsection{b)}
\section{Question 3 Solution}
\subsection{a)}
\subsection{b)}
\subsection{c)}
\subsection{d)}
\subsection{e)}
\subsection{f)}

\section{Question 4 Solution}
\subsection{a)}
\subsection{b)}
\subsection{c)}

\subsubsection{i)}

\subsubsection{ii)}
\section{Question 5 Solution}
\subsection{a)}
\subsubsection{i)}
\subsubsection{ii)}
\subsubsection{iii)}
\subsubsection{iv)}


\subsection{b)}








\end{document}

%%%%%%%%%%%%%%%%%%%%%%   EXAMPLE TABLE   %%%%%%%%%%%%%%%%%%%%%%%%%%%%%%%%
\begin{table}[H]
\begin{center}
    \caption{Resistance reading by color code convention.}
    \vspace{2mm}
    \begin{tabular}{||c | c | c||} 
        \hline
        Color Order & Value & Tolerance \\ [0.5ex] 
        \hline\hline
        Brown / Black / Red / Gold & 1k\( \Omega \) & \( \% \) 5  \\ 
        \hline
        Yellow / Violet / Red / Gold & 4.7k\( \Omega \) & \( \% \) 5   \\
        \hline
        Brown / Grey / Orange / Gold & 18k\( \Omega \) & \( \% \) 5  \\ [1ex] 
        \hline
    \end{tabular}
\end{center}
\end{table}


%%%%%%%%%%%%%%%%%%%%%%   EXAMPLE IMAGE   %%%%%%%%%%%%%%%%%%%%%%%%%%%%%%%%
\begin{figure}[H]
\centering
\includegraphics[width = 0.75\textwidth]{5.png}
\caption{Circuit schematic for the step 5}
\end{figure} 

%%%%%%%%%%%%%%%%%%%%%%   EXAMPLE IMAGE FROM PDF   %%%%%%%%%%%%%%%%%%%%%%%%%%%%%%%%
\begin{figure}[H] \centering{
	\includegraphics[scale=0.25]{2a_plot.pdf}}
	\caption{Experiment 2}
\end{figure}
