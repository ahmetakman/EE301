
\documentclass[letterpaper,12pt]{article}
\usepackage{tabularx} % extra features for tabular environment
\usepackage{amsmath}  % improve math presentation
\usepackage{float}
\usepackage{pdfpages}

\usepackage{multicol}
\usepackage{graphicx} % takes care of graphic including machinery
\graphicspath{ {./figures/} }
%\usepackage[margin=1in,letterpaper]{geometry} % decreases margins
%\usepackage{cite} % takes care of citations
\usepackage[final]{hyperref} % adds hyper links inside the generated pdf file
\hypersetup{
	colorlinks=true,       % false: boxed links; true: colored links
	linkcolor=blue,        % color of internal links
	citecolor=blue,        % color of links to bibliography
	filecolor=magenta,     % color of file links
	urlcolor =blue         
}
\usepackage[margin = 1in,headsep=0.5cm,headheight=2cm,letterpaper]{geometry} 

\usepackage{fancyhdr}
\pagestyle{fancy}
%\cfoot{center of the footer!}
\renewcommand{\headrulewidth}{0.1pt}



\begin{document}
\thispagestyle{empty}

\title{Fall 2022 EE301  \protect\\ Homework 1}
\author{Ahmet Akman 2442366 \protect\\ Hasan Oğuzhan Gök 2443083 }
\date{\today}
\maketitle
\tableofcontents
%\begin{abstract}
%abstract
%\end{abstract}
\section{Question 1 Solution}
Input/output relation: \(y_1(t)= x(t)cos(2 \pi f_0 t) \)
\\ Properties:

\begin{itemize}
    \item \textbf{Memory:} 
    \item \textbf{Linearity:}
    \item \textbf{Causality:}
    \item \textbf{Time-Invariance:}
    \item \textbf{Stability:}
\end{itemize}
Input/output relation: \(y_2(t)= c_1 x(t) + c_2 x^2(t) \)
\\ Properties:
\begin{itemize}
    \item \textbf{Memory:} 
    \item \textbf{Linearity:}
    \item \textbf{Causality:}
    \item \textbf{Time-Invariance:}
    \item \textbf{Stability:}
\end{itemize}

Input/output relation: \(y_3(t)= x(t)+4 \)
\\ Properties:
\begin{itemize}
    \item \textbf{Memory:} 
    \item \textbf{Linearity:}
    \item \textbf{Causality:}
    \item \textbf{Time-Invariance:}
    \item \textbf{Stability:}
\end{itemize}

Input/output relation: \(y_4(t)= x(t/3) \)
\\ Properties:
\begin{itemize}
    \item \textbf{Memory:} 
    \item \textbf{Linearity:}
    \item \textbf{Causality:}
    \item \textbf{Time-Invariance:}
    \item \textbf{Stability:}
\end{itemize}

Input/output relation: \(y_5(t)= t x(t+5) \)
\\ Properties:


\begin{itemize}
    \item \textbf{Memory:} 
    \item \textbf{Linearity:}
    \item \textbf{Causality:}
    \item \textbf{Time-Invariance:}
    \item \textbf{Stability:}
\end{itemize}

Input/output relation: \(y_6(t)= u(x(t)) \)
\\ Properties:
\begin{itemize}
    \item \textbf{Memory:} 
    \item \textbf{Linearity:}
    \item \textbf{Causality:}
    \item \textbf{Time-Invariance:}
    \item \textbf{Stability:}
\end{itemize}



\section{Question 2 Solution}
\subsection{a)}

\subsection{b)}
\section{Question 3 Solution}
\subsection{a)}
\subsection{b)}
\subsection{c)}
\subsection{d)}
\subsection{e)}
\subsection{f)}

\section{Question 4 Solution}
\subsection{a)}
\subsection{b)}
\subsection{c)}

\subsubsection{i)}

\subsubsection{ii)}
\section{Question 5 Solution}
\subsection{a)}
\subsubsection{i)}
\subsubsection{ii)}
\subsubsection{iii)}
\subsubsection{iv)}


\subsection{b)}








\end{document}

%%%%%%%%%%%%%%%%%%%%%%   EXAMPLE TABLE   %%%%%%%%%%%%%%%%%%%%%%%%%%%%%%%%
\begin{table}[H]
\begin{center}
    \caption{Resistance reading by color code convention.}
    \vspace{2mm}
    \begin{tabular}{||c | c | c||} 
        \hline
        Color Order & Value & Tolerance \\ [0.5ex] 
        \hline\hline
        Brown / Black / Red / Gold & 1k\( \Omega \) & \( \% \) 5  \\ 
        \hline
        Yellow / Violet / Red / Gold & 4.7k\( \Omega \) & \( \% \) 5   \\
        \hline
        Brown / Grey / Orange / Gold & 18k\( \Omega \) & \( \% \) 5  \\ [1ex] 
        \hline
    \end{tabular}
\end{center}
\end{table}


%%%%%%%%%%%%%%%%%%%%%%   EXAMPLE IMAGE   %%%%%%%%%%%%%%%%%%%%%%%%%%%%%%%%
\begin{figure}[H]
\centering
\includegraphics[width = 0.75\textwidth]{5.png}
\caption{Circuit schematic for the step 5}
\end{figure} 

%%%%%%%%%%%%%%%%%%%%%%   EXAMPLE IMAGE FROM PDF   %%%%%%%%%%%%%%%%%%%%%%%%%%%%%%%%
\begin{figure}[H] \centering{
	\includegraphics[scale=0.25]{2a_plot.pdf}}
	\caption{Experiment 2}
\end{figure}
